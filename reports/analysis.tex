\documentclass[]{article}
\makeatletter\if@twocolumn\PassOptionsToPackage{switch}{lineno}\else\fi\makeatother

%Publisher: Emerald Publishing
%Template Provided By: Typeset

\usepackage{amsmath,tabulary,graphicx,times,caption,fancyhdr,amssymb,amsfonts,amstext,amsbsy}
\usepackage[utf8]{inputenc}
\usepackage[T1]{fontenc}
\usepackage[paperheight=10in,paperwidth=6.5in,margin=2cm,headsep=.5cm,headheight=1.5cm,top=2.5cm]{geometry}
\renewenvironment{abstract} {\vspace*{-1pc}\trivlist\item[]\leftskip\oupIndent\hrulefill\par\vskip4pt\noindent\textbf{\abstractname}\mbox{\null}\\}{\par\noindent\hrulefill\endtrivlist} 
\linespread{1.13} \date{}
\captionsetup[figure]{labelfont=sc,skip=1.4pt,aboveskip=1pc}
\captionsetup[table]{labelfont=sc,skip=1.4pt,labelsep=newline}

%%%%%%%%%%%%%%%%%%%%%%%%%%%%%%%%%%%%%%%%%%%%%%%%%%%%%%%%%%%%%%%%%%%%%%%%%%
% Following additional macros are required to function some 
% functions which are not available in the class used.
%%%%%%%%%%%%%%%%%%%%%%%%%%%%%%%%%%%%%%%%%%%%%%%%%%%%%%%%%%%%%%%%%%%%%%%%%%
\usepackage{url,multirow,morefloats,floatflt,cancel,tfrupee}
\makeatletter


\AtBeginDocument{\@ifpackageloaded{textcomp}{}{\usepackage{textcomp}}}
\makeatother
\usepackage{colortbl}
\usepackage{xcolor}
\usepackage{pifont}
\usepackage[nointegrals]{wasysym}
\urlstyle{rm}
\makeatletter

%%%For Table column width calculation.
\def\mcWidth#1{\csname TY@F#1\endcsname+\tabcolsep}

%%Hacking center and right align for table
\def\cAlignHack{\rightskip\@flushglue\leftskip\@flushglue\parindent\z@\parfillskip\z@skip}
\def\rAlignHack{\rightskip\z@skip\leftskip\@flushglue \parindent\z@\parfillskip\z@skip}

%Etal definition in references
\@ifundefined{etal}{\def\etal{\textit{et~al}}}{}


%\if@twocolumn\usepackage{dblfloatfix}\fi
\usepackage{ifxetex}
\ifxetex\else\if@twocolumn\@ifpackageloaded{stfloats}{}{\usepackage{dblfloatfix}}\fi\fi

\AtBeginDocument{
\expandafter\ifx\csname eqalign\endcsname\relax
\def\eqalign#1{\null\vcenter{\def\\{\cr}\openup\jot\m@th
  \ialign{\strut$\displaystyle{##}$\hfil&$\displaystyle{{}##}$\hfil
      \crcr#1\crcr}}\,}
\fi
}

%For fixing hardfail when unicode letters appear inside table with endfloat
\AtBeginDocument{%
  \@ifpackageloaded{endfloat}%
   {\renewcommand\efloat@iwrite[1]{\immediate\expandafter\protected@write\csname efloat@post#1\endcsname{}}}{\newif\ifefloat@tables}%
}%

\def\BreakURLText#1{\@tfor\brk@tempa:=#1\do{\brk@tempa\hskip0pt}}
\let\lt=<
\let\gt=>
\def\processVert{\ifmmode|\else\textbar\fi}
\let\processvert\processVert

\@ifundefined{subparagraph}{
\def\subparagraph{\@startsection{paragraph}{5}{2\parindent}{0ex plus 0.1ex minus 0.1ex}%
{0ex}{\normalfont\small\itshape}}%
}{}

% These are now gobbled, so won't appear in the PDF.
\newcommand\role[1]{\unskip}
\newcommand\aucollab[1]{\unskip}
  
\@ifundefined{tsGraphicsScaleX}{\gdef\tsGraphicsScaleX{1}}{}
\@ifundefined{tsGraphicsScaleY}{\gdef\tsGraphicsScaleY{.9}}{}
% To automatically resize figures to fit inside the text area
\def\checkGraphicsWidth{\ifdim\Gin@nat@width>\linewidth
	\tsGraphicsScaleX\linewidth\else\Gin@nat@width\fi}

\def\checkGraphicsHeight{\ifdim\Gin@nat@height>.9\textheight
	\tsGraphicsScaleY\textheight\else\Gin@nat@height\fi}

\def\fixFloatSize#1{}%\@ifundefined{processdelayedfloats}{\setbox0=\hbox{\includegraphics{#1}}\ifnum\wd0<\columnwidth\relax\renewenvironment{figure*}{\begin{figure}}{\end{figure}}\fi}{}}
\let\ts@includegraphics\includegraphics

\def\inlinegraphic[#1]#2{{\edef\@tempa{#1}\edef\baseline@shift{\ifx\@tempa\@empty0\else#1\fi}\edef\tempZ{\the\numexpr(\numexpr(\baseline@shift*\f@size/100))}\protect\raisebox{\tempZ pt}{\ts@includegraphics{#2}}}}

%\renewcommand{\includegraphics}[1]{\ts@includegraphics[width=\checkGraphicsWidth]{#1}}
\AtBeginDocument{\def\includegraphics{\@ifnextchar[{\ts@includegraphics}{\ts@includegraphics[width=\checkGraphicsWidth,height=\checkGraphicsHeight,keepaspectratio]}}}

\DeclareMathAlphabet{\mathpzc}{OT1}{pzc}{m}{it}

\def\URL#1#2{\@ifundefined{href}{#2}{\href{#1}{#2}}}

%%For url break
\def\UrlOrds{\do\*\do\-\do\~\do\'\do\"\do\-}%
\g@addto@macro{\UrlBreaks}{\UrlOrds}



\edef\fntEncoding{\f@encoding}
\def\EUoneEnc{EU1}
\makeatother
\def\floatpagefraction{0.8} 
\def\dblfloatpagefraction{0.8}
\def\style#1#2{#2}
\def\xxxguillemotleft{\fontencoding{T1}\selectfont\guillemotleft}
\def\xxxguillemotright{\fontencoding{T1}\selectfont\guillemotright}

\newif\ifmultipleabstract\multipleabstractfalse%
\newenvironment{typesetAbstractGroup}{}{}%

%%%%%%%%%%%%%%%%%%%%%%%%%%%%%%%%%%%%%%%%%%%%%%%%%%%%%%%%%%%%%%%%%%%%%%%%%%





\usepackage[noindentafter]{titlesec}
\def\NormalBaseline{\def\baselinestretch{1.1}}

\titleformat{\section}[hang]{\NormalBaseline\filright\large\fontsize{12}{15}\bfseries\boldmath}
{\large\thesection.}
{10pt}
{\noindent}
[]
\titleformat{\subsection}[hang]{\NormalBaseline\filright\fontsize{11}{13}\bfseries\itshape\boldmath}
{\thesubsection.}
{10pt}
{}
[]
\titleformat{\subsubsection}[hang]{\NormalBaseline\filright\fontsize{10}{12}\bfseries\itshape\boldmath}
{\thesubsubsection.}
{10pt}
{}
[]
\titleformat{\paragraph}[runin]{\NormalBaseline\filright\itshape}
{\theparagraph.}
{10pt}
{}
[]
\titleformat{\subparagraph}[runin]{\NormalBaseline\filright\itshape}
{\thesubparagraph.}
{10pt}
{}
[]


\titlespacing{\section}{0pt}{1.5\baselineskip}{.2\baselineskip}  
\titlespacing{\subsection}{0pt}{1\baselineskip}{.2\baselineskip}  
\titlespacing{\subsubsection}{0pt}{1.5\baselineskip}{.2\baselineskip}  
\titlespacing{\paragraph}{0pt}{.5\baselineskip}{10pt}  
\titlespacing{\subparagraph}{0pt}{.5\baselineskip}{10pt}  


\usepackage[tablesonly]{endfloat}


\makeatletter\def\oupIndent{1pt}
\def\author#1{\gdef\@author{\hskip-\dimexpr(\tabcolsep)\hskip\oupIndent\parbox{\dimexpr\textwidth-\oupIndent}{\centering\bfseries#1}}}
\def\title#1{\gdef\@title{\centering\bfseries\ifx\@articleType\@empty\else\@articleType\\\fi#1}}
\let\@articleType\@empty \def\articletype#1{\gdef\@articleType{{\normalfont\itshape#1}}}
\fancypagestyle{headings}{\fancyhf{}\fancyhead[C]{\RunningHead\hspace*{1pc}}\fancyhead[R]{\thepage}}\pagestyle{headings}
\emergencystretch =15pt 
\makeatother
\usepackage[authoryear, round]{natbib} % authoryear for author-year citations, round for parentheses
\usepackage[hidelinks]{hyperref}

\begin{document}


\title{Lean JIT and Environmental Performance: an empirical analysis}
\author{Alessa Aila\textsuperscript{1}\thanks{E-mail: alessa.aila@aalto.fi}{ },
            Astrid Holstr{\"{o}}m\textsuperscript{1}\thanks{E-mail: astrid.holmstrom@aalto.fi}{ },
            Eemil Rantala\textsuperscript{1}\thanks{E-mail: eemil.rantala@aalto.fi}{ },
            John Anderson\textsuperscript{1}\thanks{E-mail: john.anderson@aalto.fi}{ } and
            Valtteri Luodem{\"{a}}ki\textsuperscript{1}\thanks{E-mail: valtteri.luodemaki@aalto.fi}{ }~\\[-3pt]\normalsize\normalfont\itshape 
~\\\textsuperscript{1}{Department of Industrial Engineering and Management\unskip, Aalto University}}
\def\RunningHead{{Lean JIT and Environmental Performance: an empirical analysis}}

\maketitle 


\begin{abstract}
TBA
\end{abstract}\def\keywordstitle{Keywords}
    
\section{Introduction}
Over the past decade there has been an increase in the research published on the synergies and trade-offs between lean manufacturing and environmental performance \citep{henaoLeanManufacturingSustainable2019, abualfaraaLeanGreenManufacturingPractices2020, diesteRelationshipLeanEnvironmental2019, lobomesquitaExploringRelationshipsIntegrating2022, garza-reyesLeanGreenSystematic2015, kingLeanGreenEmpirical2009}.

These combined approaches, often dubbed `lean-green', typically cites the Triple-Bottom-Line concept, which postulates the need for performance in economic growth, environmental preservation, and social responsibility, in order to achieve sustainability \citep{henaoLeanManufacturingSustainable2019}. 
Motivated by this body of research as well as our interest in sustainability studies, we have decided to study the effect of environmental and lean practices on environmental performance. 

Abualfaraa et al. outline several research gaps and opportunities for those interested in lean-green manufacturing. 
In their Structured Literature Review of articles published between 2000 and 2018, they have identified several research directions in both the synergies and incompatibilities between environmental and lean practices \citep{abualfaraaLeanGreenManufacturingPractices2020}. 
On one line, it is argued that lean practices may work as a catalyst for environmental practices and innovation through its focus on waste reduction and continuous improvement.
On the other, the incompatibilities between the two approaches are also studied. 
Just in time (JIT) practices have been specifically highlighted. 
For example JIT manufacturing practices such as small lot sizes and high replenishment frequency implies more frequent transportation, higher CO2 emissions and more packaging waste \citep{diesteRelationshipLeanEnvironmental2019}.

Literature reviews also pointed out the need for more quantitative research with a focus on robust, well-defined sustainability metrics \citep{abualfaraaLeanGreenManufacturingPractices2020}. 
Through an empirical analysis of Lean JIT and environmental practices, our goal is to contribute to this research agenda.
    
\section{Literature Review}
WIP

RQ 1: What effect does JIT practices have on environmental performance?

RQ 2: What are the combined effects of JIT and environmental practices on environmental performance?

RQ 3: What is the effect of JIT practices on C02 emissions and packaging waste?

\section{Hypothesis}
WIP

H1: JIT practices are positively correlated with overall environmental performance.

H2: JIT practices are negativly correlated with reducing CO2 emissions and packaging waste.

H3: JIT practices negativly moderate the effect of environmental practices on reducing CO2 emissions and packaging waste.

H4: Environmental practices are JIT practices are complementary: the implementation of JIT practices increases the marginal return of environmental practices on overall environmental performance and vice versa.

\section{Methods}
WIP 

Exploratory factor analysis

Confirmatory factor analysis

Moderating effect of lean JIT 

Complimentarity of lean JIT/ environemntal practices

\section{Results}
WIP

\section{Discussion}
WIP

Non english speaking literature often filtered out

Diffirent national contexts

Different industries

Critique of tripple bottom line is lack of novelty around the hardest problem, social sustainability

Solutions to the JIT/Green dilemma: They suggest that this can be done by, for example, selecting suppliers from a certain geographic area to enable truckload sharing for delivering or, when small amounts have to be delivered, managing the routes in order to supply multiple customers in the same area. 

TO-DO:

- Create environemental practice "bundles" using EFA

- Check for complimentarity between JIT and EP bundles as per complimentarity paper approaches

- Test moderating effect of JIT on EP bundles as per china paper

- Apply spell check to latek sections before submitting
\bibliographystyle{plainnat} % This style is compatible with natbib and author-year
\bibliography{\jobname}

\end{document}