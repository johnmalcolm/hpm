\documentclass[]{article}
\makeatletter\if@twocolumn\PassOptionsToPackage{switch}{lineno}\else\fi\makeatother

%Publisher: Emerald Publishing
%Template Provided By: Typeset

\usepackage{amsmath,tabulary,graphicx,times,caption,fancyhdr,amssymb,amsfonts,amstext,amsbsy}
\usepackage[utf8]{inputenc}
\usepackage[T1]{fontenc}
\usepackage[paperheight=10in,paperwidth=6.5in,margin=2cm,headsep=.5cm,headheight=1.5cm,top=2.5cm]{geometry}
\renewenvironment{abstract} {\vspace*{-1pc}\trivlist\item[]\leftskip\oupIndent\hrulefill\par\vskip4pt\noindent\textbf{\abstractname}\mbox{\null}\\}{\par\noindent\hrulefill\endtrivlist} 
\linespread{1.13} \date{}
\captionsetup[figure]{labelfont=sc,skip=1.4pt,aboveskip=1pc}
\captionsetup[table]{labelfont=sc,skip=1.4pt,labelsep=newline}

%%%%%%%%%%%%%%%%%%%%%%%%%%%%%%%%%%%%%%%%%%%%%%%%%%%%%%%%%%%%%%%%%%%%%%%%%%
% Following additional macros are required to function some 
% functions which are not available in the class used.
%%%%%%%%%%%%%%%%%%%%%%%%%%%%%%%%%%%%%%%%%%%%%%%%%%%%%%%%%%%%%%%%%%%%%%%%%%
\usepackage{url,multirow,morefloats,floatflt,cancel,tfrupee}
\makeatletter


\AtBeginDocument{\@ifpackageloaded{textcomp}{}{\usepackage{textcomp}}}
\makeatother
\usepackage{colortbl}
\usepackage{xcolor}
\usepackage{pifont}
\usepackage{longtable}  % For long tables spanning multiple pages
\usepackage{booktabs}   % For top, mid and bottom rules
\usepackage{caption}  % For captions in floating environments
\usepackage{array}      % For table column formatting
\usepackage{geometry}   % Optional: For setting page margins
\usepackage{pdflscape}  % For the landscape environment
\usepackage{microtype}
\usepackage{float}
\usepackage[nointegrals]{wasysym}
\urlstyle{rm}
\makeatletter

%%%For Table column width calculation.
\def\mcWidth#1{\csname TY@F#1\endcsname+\tabcolsep}

%%Hacking center and right align for table
\def\cAlignHack{\rightskip\@flushglue\leftskip\@flushglue\parindent\z@\parfillskip\z@skip}
\def\rAlignHack{\rightskip\z@skip\leftskip\@flushglue \parindent\z@\parfillskip\z@skip}

%Etal definition in references
\@ifundefined{etal}{\def\etal{\textit{et~al}}}{}


%\if@twocolumn\usepackage{dblfloatfix}\fi
\usepackage{ifxetex}
\ifxetex\else\if@twocolumn\@ifpackageloaded{stfloats}{}{\usepackage{dblfloatfix}}\fi\fi

\AtBeginDocument{
\expandafter\ifx\csname eqalign\endcsname\relax
\def\eqalign#1{\null\vcenter{\def\\{\cr}\openup\jot\m@th
  \ialign{\strut$\displaystyle{##}$\hfil&$\displaystyle{{}##}$\hfil
      \crcr#1\crcr}}\,}
\fi
}

%For fixing hardfail when unicode letters appear inside table with endfloat
\AtBeginDocument{%
  \@ifpackageloaded{endfloat}%
   {\renewcommand\efloat@iwrite[1]{\immediate\expandafter\protected@write\csname efloat@post#1\endcsname{}}}{\newif\ifefloat@tables}%
}%

\def\BreakURLText#1{\@tfor\brk@tempa:=#1\do{\brk@tempa\hskip0pt}}
\let\lt=<
\let\gt=>
\def\processVert{\ifmmode|\else\textbar\fi}
\let\processvert\processVert

\@ifundefined{subparagraph}{
\def\subparagraph{\@startsection{paragraph}{5}{2\parindent}{0ex plus 0.1ex minus 0.1ex}%
{0ex}{\normalfont\small\itshape}}%
}{}

% These are now gobbled, so won't appear in the PDF.
\newcommand\role[1]{\unskip}
\newcommand\aucollab[1]{\unskip}
  
\@ifundefined{tsGraphicsScaleX}{\gdef\tsGraphicsScaleX{1}}{}
\@ifundefined{tsGraphicsScaleY}{\gdef\tsGraphicsScaleY{.9}}{}
% To automatically resize figures to fit inside the text area
\def\checkGraphicsWidth{\ifdim\Gin@nat@width>\linewidth
	\tsGraphicsScaleX\linewidth\else\Gin@nat@width\fi}

\def\checkGraphicsHeight{\ifdim\Gin@nat@height>.9\textheight
	\tsGraphicsScaleY\textheight\else\Gin@nat@height\fi}

\def\fixFloatSize#1{}%\@ifundefined{processdelayedfloats}{\setbox0=\hbox{\includegraphics{#1}}\ifnum\wd0<\columnwidth\relax\renewenvironment{figure*}{\begin{figure}}{\end{figure}}\fi}{}}
\let\ts@includegraphics\includegraphics

\def\inlinegraphic[#1]#2{{\edef\@tempa{#1}\edef\baseline@shift{\ifx\@tempa\@empty0\else#1\fi}\edef\tempZ{\the\numexpr(\numexpr(\baseline@shift*\f@size/100))}\protect\raisebox{\tempZ pt}{\ts@includegraphics{#2}}}}

%\renewcommand{\includegraphics}[1]{\ts@includegraphics[width=\checkGraphicsWidth]{#1}}
\AtBeginDocument{\def\includegraphics{\@ifnextchar[{\ts@includegraphics}{\ts@includegraphics[width=\checkGraphicsWidth,height=\checkGraphicsHeight,keepaspectratio]}}}

\DeclareMathAlphabet{\mathpzc}{OT1}{pzc}{m}{it}

\def\URL#1#2{\@ifundefined{href}{#2}{\href{#1}{#2}}}

%%For url break
\def\UrlOrds{\do\*\do\-\do\~\do\'\do\"\do\-}%
\g@addto@macro{\UrlBreaks}{\UrlOrds}



\edef\fntEncoding{\f@encoding}
\def\EUoneEnc{EU1}
\makeatother
\def\floatpagefraction{0.8} 
\def\dblfloatpagefraction{0.8}
\def\style#1#2{#2}
\def\xxxguillemotleft{\fontencoding{T1}\selectfont\guillemotleft}
\def\xxxguillemotright{\fontencoding{T1}\selectfont\guillemotright}

\newif\ifmultipleabstract\multipleabstractfalse%
\newenvironment{typesetAbstractGroup}{}{}%

%%%%%%%%%%%%%%%%%%%%%%%%%%%%%%%%%%%%%%%%%%%%%%%%%%%%%%%%%%%%%%%%%%%%%%%%%%





\usepackage[noindentafter]{titlesec}
\def\NormalBaseline{\def\baselinestretch{1.1}}

\titleformat{\section}[hang]{\NormalBaseline\filright\large\fontsize{12}{15}\bfseries\boldmath}
{\large\thesection.}
{10pt}
{\noindent}
[]
\titleformat{\subsection}[hang]{\NormalBaseline\filright\fontsize{11}{13}\bfseries\itshape\boldmath}
{\thesubsection.}
{10pt}
{}
[]
\titleformat{\subsubsection}[hang]{\NormalBaseline\filright\fontsize{10}{12}\bfseries\itshape\boldmath}
{\thesubsubsection.}
{10pt}
{}
[]
\titleformat{\paragraph}[runin]{\NormalBaseline\filright\itshape}
{\theparagraph.}
{10pt}
{}
[]
\titleformat{\subparagraph}[runin]{\NormalBaseline\filright\itshape}
{\thesubparagraph.}
{10pt}
{}
[]


\titlespacing{\section}{0pt}{1.5\baselineskip}{.2\baselineskip}  
\titlespacing{\subsection}{0pt}{1\baselineskip}{.2\baselineskip}  
\titlespacing{\subsubsection}{0pt}{1.5\baselineskip}{.2\baselineskip}  
\titlespacing{\paragraph}{0pt}{.5\baselineskip}{10pt}  
\titlespacing{\subparagraph}{0pt}{.5\baselineskip}{10pt}  



\makeatletter\def\oupIndent{1pt}
\def\author#1{\gdef\@author{\hskip-\dimexpr(\tabcolsep)\hskip\oupIndent\parbox{\dimexpr\textwidth-\oupIndent}{\centering\bfseries#1}}}
\def\title#1{\gdef\@title{\centering\bfseries\ifx\@articleType\@empty\else\@articleType\\\fi#1}}
\let\@articleType\@empty \def\articletype#1{\gdef\@articleType{{\normalfont\itshape#1}}}
\fancypagestyle{headings}{\fancyhf{}\fancyhead[C]{\RunningHead\hspace*{1pc}}\fancyhead[R]{\thepage}}\pagestyle{headings}
\emergencystretch =15pt 
\makeatother
\usepackage[authoryear, round]{natbib} % authoryear for author-year citations, round for parentheses
\usepackage[hidelinks]{hyperref}

\begin{document}


\title{JIT and Environmental Performance: an empirical analysis}
\author{Alessa Aila\textsuperscript{1}\thanks{E-mail: alessa.aila@aalto.fi}{ },
            Astrid Holstr{\"{o}}m\textsuperscript{1}\thanks{E-mail: astrid.holmstrom@aalto.fi}{ },
            Eemil Rantala\textsuperscript{1}\thanks{E-mail: eemil.rantala@aalto.fi}{ },
            John Anderson\textsuperscript{1}\thanks{E-mail: john.anderson@aalto.fi}{ } and
            Valtteri Luodem{\"{a}}ki\textsuperscript{1}\thanks{E-mail: valtteri.luodemaki@aalto.fi}{ }~\\[-3pt]\normalsize\normalfont\itshape 
~\\\textsuperscript{1}{Department of Industrial Engineering and Management\unskip, Aalto University}}
\def\RunningHead{{Lean JIT and Environmental Performance: an empirical analysis}}

\maketitle 


\begin{abstract}
TBA
\end{abstract}\def\keywordstitle{Keywords}
    
\section{Introduction}
Over the past decade there has been an increase in the research published on the synergies and trade-offs between lean manufacturing and environmental performance \citep{henaoLeanManufacturingSustainable2019, abualfaraaLeanGreenManufacturingPractices2020, diesteRelationshipLeanEnvironmental2019, lobomesquitaExploringRelationshipsIntegrating2022, garza-reyesLeanGreenSystematic2015, kingLeanGreenEmpirical2009}.

These combined approaches, often dubbed `lean-green', typically cites the Triple-Bottom-Line concept, which postulates the need for performance in economic growth, environmental preservation, and social responsibility, in order to achieve sustainability \citep{henaoLeanManufacturingSustainable2019}. 
Motivated by this body of research as well as our interest in sustainability studies, we have decided to study the effect of environmental and lean practices on environmental performance. 

Abualfaraa et al. outline several research gaps and opportunities for those interested in lean-green manufacturing. 
In their Structured Literature Review of articles published between 2000 and 2018, they have identified several research directions in both the synergies and incompatibilities between environmental and lean practices \citep{abualfaraaLeanGreenManufacturingPractices2020}. 
On one line, it is argued that lean practices may work as a catalyst for environmental practices and innovation through its focus on waste reduction and continuous improvement.
On the other, the incompatibilities between the two approaches are also studied. 
Just in time (JIT) practices have been specifically highlighted. 
For example JIT manufacturing practices such as small lot sizes and high replenishment frequency implies more frequent transportation, higher CO2 emissions and more packaging waste \citep{diesteRelationshipLeanEnvironmental2019}.

Literature reviews also pointed out the need for more quantitative research with a focus on robust, well-defined sustainability metrics \citep{abualfaraaLeanGreenManufacturingPractices2020}. 
Through an empirical analysis of JIT and environmental practices, our goal is to contribute to this research agenda.
    
\section{Literature Review}
WIP

RQ 1: What effect do lean JIT practices have on environmental practices and environmental performance?

RQ 2: What is the effect of JIT practices on C02 emissions and packaging waste?

\section{Hypothesis}
WIP

H1: Environmental practices and JIT practices are complementary: the implementation of JIT practices increases the marginal return of environmental practices on environmental performance and vice versa.

H2: JIT practices negatively moderates the effect of environmental practices on emissions to air and solid waste generation.

\section{Methods}
WIP  

Confirmatory factor analysis for establishing legitamacy of 3 bundles

Complimentarity of lean JIT/ environemntal practices

Moderating effect of lean JIT 

\begin{landscape}
\small
\begin{longtable}{l@{\hspace{6pt}}l@{\hspace{6pt}}p{11cm}@{\hspace{6pt}}l@{\hspace{6pt}}l@{\hspace{6pt}}l}
\caption{Confirmatory Factor Analysis} \label{tab:your_label} \\
\toprule
Bundle & HPM Code & Item Description & Loading & SE & t-value \\
\midrule
\endfirsthead
\caption[]{Confirmatory Factor Analysis} \\
\toprule
Bundle & HPM Code & Item Description & Loading & SE & t-value \\
\midrule
\endhead
\midrule
\multicolumn{6}{r}{Continued on next page} \\
\midrule
\endfoot
\bottomrule
\endlastfoot
Environmental Practices & ENVRTX21 & Environmentally preferable packaging for the products that you produce (recycled content, less volume, reusable packaging) & 0.63*** & 0.06 & 11.11 \\
 & ENVRTX37 & Using a third party to monitor working conditions at supplier facilities & 0.8*** & 0.08 & 9.75 \\
 & ENVRTX02 & Water efficiency & 0.88*** & 0.07 & 12.97 \\
 & ENVRTX22 & Substituting environmental preferable direct materials or supplies for harmful or non-renewable ones & 0.69*** & 0.06 & 11.44 \\
 & ENVRTX39 & Providing design specification to suppliers in line with environmental requirements (e.g. green purchasing, black list of raw materials) & 1.06*** & 0.08 & 13.5 \\
 & ENVRTX23 & Environmental improvements in the disposition of your organization’s scrap or excess material (re-use, recycling, etc.) & 0.58*** & 0.05 & 10.82 \\
 & ENVRTX18 & Working with customers to help them achieve environmental objectives & 1.12*** & 0.08 & 14.25 \\
 & ENVRTX13 & Complying with a customer’s supplier code of conduct & 0.91*** & 0.08 & 11.8 \\
 & ENVRTX33 & Starting or maintaining a formal M/WBE supplier purchase program & 1.0*** & 0.08 & 11.79 \\
 & ENVRTX03 & Reducing waste in internal processes (e.g., improving yield or efficiency) & 0.61*** & 0.05 & 11.79 \\
 & ENVRTX20 & Life-cycle analysis of the “cradle to grave” environmental impact of materials/products & 1.19*** & 0.08 & 14.62 \\
 & ENVRTX38 & Incorporating environmental considerations in evaluating and selecting suppliers & 1.16*** & 0.07 & 15.53 \\
 & ENVRTX08 & Decreasing the likelihood or impact of an environmental accident & 0.67*** & 0.05 & 12.3 \\
 & ENVRTX05 & Pollution prevention (eliminating emissions or waste) & 0.72*** & 0.06 & 12.89 \\
 & ENVRTX30 & Giving preference to materials with third party certifications, such as Green Seal, FSC or Energy Star & 1.02*** & 0.08 & 13.2 \\
 & ENVRTX24 & Environmental improvements in the disposition of your organization’s equipment & 0.97*** & 0.06 & 15.36 \\
 & ENVRTX32 & Purchasing from minority- or women-owned business enterprise (M/WBE) suppliers & 0.98*** & 0.08 & 12.96 \\
 & ENVRTX34 & Visiting suppliers’ plants or ensuring that they are not using sweatshop labor & 1.07*** & 0.09 & 12.5 \\
 & ENVRTX04 & Improving the workforce environment (e.g., indoor air quality) & 0.57*** & 0.05 & 11.27 \\
 & ENVRTX29 & Encouraging suppliers to improve the environmental performance of their processes & 1.28*** & 0.08 & 16.88 \\
 & ENVRTX41 & Involvement of suppliers in the re-design of internal processes (e.g. remanufacturing, reduction of by-products) & 1.02*** & 0.07 & 14.73 \\
 & ENVRTX40 & Co-development with suppliers to reduce the environmental impact of the product (e.g. eco-design, green packaging, recyclability) & 1.09*** & 0.07 & 15.24 \\
 & ENVRTX09 & Reduction/avoidance of land consumption & 1.13*** & 0.09 & 13.24 \\
 & ENVRTX17 & Carbon tracking/carbon footprint calculation of supply chain & 1.11*** & 0.09 & 12.68 \\
 & ENVRTX07 & Remediation projects, such as cleanup or restoration from past practices & 1.18*** & 0.09 & 12.57 \\
 & ENVRTX11 & Improvements in outbound transportation, such as fuel efficiency or load matching & 1.12*** & 0.08 & 14.1 \\
 & ENVRTX10 & Improvements in inbound transportation, such as fuel efficiency or load matching & 1.1*** & 0.08 & 14.38 \\
 & ENVRTX01 & Energy efficiency or renewable energy & 0.77*** & 0.07 & 11.55 \\
 & ENVRTX14 & Complying with an industry-wide code of conduct & 0.87*** & 0.06 & 14.21 \\
 & ENVRTX15 & Other compliance or auditing program focused on your plant (not on your suppliers) & 0.88*** & 0.06 & 13.72 \\
 & ENVRTX12 & Seeking or maintaining ISO14001 certification & 0.85*** & 0.09 & 9.73 \\
 & ENVRTX31 & Requesting that your suppliers sign a code of environmental conduct & 1.16*** & 0.09 & 12.72 \\
 & ENVRTX35 & Ensuring that suppliers comply with child labor laws & 1.12*** & 0.1 & 11.7 \\
 & ENVRTX36 & Asking suppliers to pay a “living wage” & 1.04*** & 0.09 & 11.17 \\
 & ENVRTX06 & Pollution control (scrubbing, waste treatment) & 0.76*** & 0.07 & 11.27 \\
 & EPRACX01 & Implementation of a certified environmental management system, such as ISO 14000. & 0.96*** & 0.09 & 10.3 \\
 & EPRACX02 & Implementation of internal environmental management procedures (e.g. environmental training program, internal environmental audit, newsletter). & 0.96*** & 0.08 & 12.34 \\
 & EPRACX03 & Use of cleaner technologies in the production process (e.g. abatement equipment) to reduce pollution emissions and/or resource use. & 0.98*** & 0.07 & 14.2 \\
 & EPRACX04 & Environment-friendly product design. & 1.21*** & 0.08 & 15.58 \\
 & EPRACX05 & Environmental improvement of packaging. & 1.0*** & 0.07 & 14.74 \\
 & EPRACX06 & Use of environment-friendly raw materials. & 0.99*** & 0.07 & 14.6 \\
JIT Practices & LAYOUTN01 & We have laid out the shop floor so that processes and machines are in close proximity to each other. & 0.71*** & 0.06 & 11.66 \\
 & LAYOUTN02 & The layout of our shop floor facilitates low inventories and fast throughput. & 0.79*** & 0.07 & 12.04 \\
 & LAYOUTN03 & Our processes are located close together, so that material handling and part storage are minimized. & 0.88*** & 0.07 & 11.87 \\
 & LAYOUTN04 & We have located our machines to support JIT production flow. & 1.03*** & 0.08 & 13.77 \\
 & JITDELN01 & Our suppliers deliver to us on a just-in-time basis. & 1.09*** & 0.09 & 12.74 \\
 & JITDELN02 & We receive daily shipments from most suppliers. & 0.8*** & 0.09 & 9.13 \\
 & JITDELN03 & Our suppliers are linked with us by a pull system. & 1.1*** & 0.09 & 12.2 \\
 & KANBANN01 & Suppliers fill our kanban containers, rather than filling purchase orders. & 0.73*** & 0.09 & 8.39 \\
 & KANBANN02 & We use a kanban pull system for production control. & 1.05*** & 0.09 & 11.34 \\
 & KANBANN03 & We use kanban squares, containers or signals for production control. & 1.08*** & 0.09 & 11.56 \\
 & LINKCN01 & Our customers receive just-in-time deliveries from us. & 1.04*** & 0.08 & 12.82 \\
 & LINKCN02 & We always deliver on time to our customers. & 0.71*** & 0.06 & 10.97 \\
 & LINKCN03 & We can adapt our production schedule to sudden production stoppages by our customers. & 0.77*** & 0.07 & 11.39 \\
 & LINKCN04 & Our customers have a pull type link with us. & 1.18*** & 0.09 & 12.66 \\
 & LINKCN05 & Our customers are linked with us via JIT systems. & 1.24*** & 0.09 & 13.44 \\
 & SCHEDN01 & We usually meet the production schedule each day. & 0.75*** & 0.06 & 12.46 \\
 & SCHEDN02 & We usually complete our daily schedule as planned. & 0.68*** & 0.05 & 12.59 \\
 & SETUPN01 & We are aggressively working to lower setup times in our plant. & 0.76*** & 0.07 & 10.88 \\
 & SETUPN02 & We have low setup times of equipment in our plant. & 0.81*** & 0.07 & 11.53 \\
 & SETUPN03 & Our workers practice setups, in order to reduce the time required. & 1.04*** & 0.09 & 11.84 \\
Environmental Performance & EPERFX01 & Overall environmental performance. & 0.83*** & 0.06 & 14.96 \\
 & EPERFX02 & Raw materials consumption. & 0.77*** & 0.05 & 14.78 \\
 & EPERFX03 & Energy consumption. & 0.96*** & 0.06 & 16.74 \\
 & EPERFX04 & Water consumption. & 0.94*** & 0.06 & 17.02 \\
 & EPERFX05 & Emissions to air. & 0.89*** & 0.06 & 15.69 \\
 & EPERFX06 & Releases to water. & 0.81*** & 0.06 & 14.38 \\
 & EPERFX07 & Solid waste generation (e.g. landfill capacity consumed). & 0.7*** & 0.05 & 13.53 \\
 & EPERFX08 & Waste recovery (e.g. recycling). & 0.59*** & 0.05 & 11.7 \\
 & EPERFX09 & Fines or other violations of environmental rules/regulations. & 0.84*** & 0.07 & 11.57 \\
\end{longtable}

\end{landscape}
\section{Results}
\begin{table}[htbp]
    \centering
    \caption{Frequency of Adoption and Environmental Performance}
    \label{tab:your_label}
    \begin{tabular}{lrll}
\toprule
Category & Frequency & Percentage & Mean of Performance \\
\midrule
High JIT \& Environmental & 57 & 32.57 & 3.91 \\
Mainly Environmental & 38 & 21.71 & 3.78 \\
Mainly JIT & 40 & 22.86 & 3.58 \\
Low JIT \& Environmental & 40 & 22.86 & 3.28 \\
\bottomrule
\end{tabular}

    \end{table}
    
adf ba 
\section{Discussion}
WIP

Non english speaking literature often filtered out

Diffirent national contexts

Different industries

Critique of tripple bottom line is lack of novelty around the hardest problem, social sustainability

Solutions to the JIT/Green dilemma: They suggest that this can be done by, for example, selecting suppliers from a certain geographic area to enable truckload sharing for delivering or, when small amounts have to be delivered, managing the routes in order to supply multiple customers in the same area. 

BACKLOG:

- Create skeleton for Lit Review with article references and share with Alessa and Astrdid

- Create environemental practice "bundles" using EFA (DONE)

- Run a CFA between JIT and EP
 - Understand how to properly deal with NA values in CFA

- Run correlations

- Check for complimentarity between JIT and EP bundles as per complimentarity paper approaches

- See if there is a better way then just dropping all NA rows

- Test moderating effect of JIT on EP bundles as per china paper

- We need to ask the specifics of the likert scale for the JIT measures

- Apply spell check to latek sections before submitting
\bibliographystyle{plainnat} % This style is compatible with natbib and author-year
\bibliography{\jobname}

\end{document}