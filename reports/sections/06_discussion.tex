\section*{Discussion}
\subsection{Complimentarity \& Moderating Effects}
We were suprised to find that there was no statistically significant difference between the means for High JIT \& High Env Practices and Mainly Environmental.
This suggests that there is no complimentarity between JIT and environmental practices.
There was also no statistically significant moderating effect or individual effect of JIT on the relationship between environmental practices and environmental performance.
It is also worth noting that there should be more robustness checks done on the regression analysis.
This could include checking for multicollinearity and checking for heteroskedasticity.
\\
\\
In the future we would like to run a VIF test specifically and to look for correlations between the independent variables.
There is also some questions about the control variables needed for the regression analysis. 
\subsection{Findings \& Limitations}
The lack of complimentarity does not suggest a negative relationship between JIT and environmental performance but it does suggest that there is no marginal gains from the combination of JIT and environmental practices on our environmental performance bundle.
\\
\\
The way we have formed our bundles in the EFA may have contributed to this result, but it is worth noting that the outcomes were the same for our Hypothesis when running a CFA on the full set of practices from the relevant HPM round 4 scales (see appendix).
The question of JITs impact on environmental performance is still an open one.
We suggest more work to be done both on the analysis but also on the data collection that can support further research on the specific relationship between JIT and environmental performance.
It may also be useful to go back the literature to help identify the most salient independant variables related to JITs environmental effects, in order to develop a more informed model.