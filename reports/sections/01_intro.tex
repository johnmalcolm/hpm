\section{Introduction}
Over the past decade there has been an increase in the research published on the synergies and trade-offs between lean manufacturing and environmental performance \citep{henaoLeanManufacturingSustainable2019, abualfaraaLeanGreenManufacturingPractices2020, diesteRelationshipLeanEnvironmental2019, lobomesquitaExploringRelationshipsIntegrating2022, garza-reyesLeanGreenSystematic2015, kingLeanGreenEmpirical2009}.

These combined approaches, often dubbed `lean-green', typically cites the Triple-Bottom-Line concept, which postulates the need for performance in economic growth, environmental preservation, and social responsibility, in order to achieve sustainability \citep{henaoLeanManufacturingSustainable2019}. 
Motivated by this body of research as well as our interest in sustainability studies, we have decided to study the effect of environmental and lean practices on environmental performance. 

Abualfaraa et al. outline several research gaps and opportunities for those interested in lean-green manufacturing. 
In their Structured Literature Review of articles published between 2000 and 2018, they have identified several research directions in both the synergies and incompatibilities between environmental and lean practices \citep{abualfaraaLeanGreenManufacturingPractices2020}. 
On one line, it is argued that lean practices may work as a catalyst for environmental practices and innovation through its focus on waste reduction and continuous improvement.
On the other, the incompatibilities between the two approaches are also studied. 
Just in time (JIT) practices have been specifically highlighted. 
For example JIT manufacturing practices such as small lot sizes and high replenishment frequency implies more frequent transportation, higher CO2 emissions and more packaging waste \citep{diesteRelationshipLeanEnvironmental2019}.

Literature reviews also pointed out the need for more quantitative research with a focus on robust, well-defined sustainability metrics \citep{abualfaraaLeanGreenManufacturingPractices2020}. 
Through an empirical analysis of JIT and environmental practices, our goal is to contribute to this research agenda.