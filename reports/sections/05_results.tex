\section{Results}
\subsection*{Confirmatory Factor Analysis}
Preliminary Discussion:

- In the base article \cite{furlanComplementarityLeanManufacturing2011}, they have only six practices for each bundle.

- Should we create subsamples to get down to six per factor, or is it okay to include everything?

- The majority of loadings are above 0.8, and all statistically significant (p < 0.01).

- We tried bundling sustainability outcomes at some stage too, but in the end, it seemed irrelevant to our research questions.

- We also conducted an EFA to see if we could further breakdown the bundles (specifically environmental practices, which is a massive group).

- We did not include EFA results here as it did not seem relevant to the research questions (should we include it or better use this time to diagnose our other models).

% \begin{landscape}
\small
\begin{longtable}{l@{\hspace{6pt}}l@{\hspace{6pt}}p{11cm}@{\hspace{6pt}}l@{\hspace{6pt}}l@{\hspace{6pt}}l}
\caption{Confirmatory Factor Analysis} \label{tab:your_label} \\
\toprule
Bundle & HPM Code & Item Description & Loading & SE & t-value \\
\midrule
\endfirsthead
\caption[]{Confirmatory Factor Analysis} \\
\toprule
Bundle & HPM Code & Item Description & Loading & SE & t-value \\
\midrule
\endhead
\midrule
\multicolumn{6}{r}{Continued on next page} \\
\midrule
\endfoot
\bottomrule
\endlastfoot
Environmental Practices & ENVRTX21 & Environmentally preferable packaging for the products that you produce (recycled content, less volume, reusable packaging) & 0.63*** & 0.06 & 11.11 \\
 & ENVRTX37 & Using a third party to monitor working conditions at supplier facilities & 0.8*** & 0.08 & 9.75 \\
 & ENVRTX02 & Water efficiency & 0.88*** & 0.07 & 12.97 \\
 & ENVRTX22 & Substituting environmental preferable direct materials or supplies for harmful or non-renewable ones & 0.69*** & 0.06 & 11.44 \\
 & ENVRTX39 & Providing design specification to suppliers in line with environmental requirements (e.g. green purchasing, black list of raw materials) & 1.06*** & 0.08 & 13.5 \\
 & ENVRTX23 & Environmental improvements in the disposition of your organization’s scrap or excess material (re-use, recycling, etc.) & 0.58*** & 0.05 & 10.82 \\
 & ENVRTX18 & Working with customers to help them achieve environmental objectives & 1.12*** & 0.08 & 14.25 \\
 & ENVRTX13 & Complying with a customer’s supplier code of conduct & 0.91*** & 0.08 & 11.8 \\
 & ENVRTX33 & Starting or maintaining a formal M/WBE supplier purchase program & 1.0*** & 0.08 & 11.79 \\
 & ENVRTX03 & Reducing waste in internal processes (e.g., improving yield or efficiency) & 0.61*** & 0.05 & 11.79 \\
 & ENVRTX20 & Life-cycle analysis of the “cradle to grave” environmental impact of materials/products & 1.19*** & 0.08 & 14.62 \\
 & ENVRTX38 & Incorporating environmental considerations in evaluating and selecting suppliers & 1.16*** & 0.07 & 15.53 \\
 & ENVRTX08 & Decreasing the likelihood or impact of an environmental accident & 0.67*** & 0.05 & 12.3 \\
 & ENVRTX05 & Pollution prevention (eliminating emissions or waste) & 0.72*** & 0.06 & 12.89 \\
 & ENVRTX30 & Giving preference to materials with third party certifications, such as Green Seal, FSC or Energy Star & 1.02*** & 0.08 & 13.2 \\
 & ENVRTX24 & Environmental improvements in the disposition of your organization’s equipment & 0.97*** & 0.06 & 15.36 \\
 & ENVRTX32 & Purchasing from minority- or women-owned business enterprise (M/WBE) suppliers & 0.98*** & 0.08 & 12.96 \\
 & ENVRTX34 & Visiting suppliers’ plants or ensuring that they are not using sweatshop labor & 1.07*** & 0.09 & 12.5 \\
 & ENVRTX04 & Improving the workforce environment (e.g., indoor air quality) & 0.57*** & 0.05 & 11.27 \\
 & ENVRTX29 & Encouraging suppliers to improve the environmental performance of their processes & 1.28*** & 0.08 & 16.88 \\
 & ENVRTX41 & Involvement of suppliers in the re-design of internal processes (e.g. remanufacturing, reduction of by-products) & 1.02*** & 0.07 & 14.73 \\
 & ENVRTX40 & Co-development with suppliers to reduce the environmental impact of the product (e.g. eco-design, green packaging, recyclability) & 1.09*** & 0.07 & 15.24 \\
 & ENVRTX09 & Reduction/avoidance of land consumption & 1.13*** & 0.09 & 13.24 \\
 & ENVRTX17 & Carbon tracking/carbon footprint calculation of supply chain & 1.11*** & 0.09 & 12.68 \\
 & ENVRTX07 & Remediation projects, such as cleanup or restoration from past practices & 1.18*** & 0.09 & 12.57 \\
 & ENVRTX11 & Improvements in outbound transportation, such as fuel efficiency or load matching & 1.12*** & 0.08 & 14.1 \\
 & ENVRTX10 & Improvements in inbound transportation, such as fuel efficiency or load matching & 1.1*** & 0.08 & 14.38 \\
 & ENVRTX01 & Energy efficiency or renewable energy & 0.77*** & 0.07 & 11.55 \\
 & ENVRTX14 & Complying with an industry-wide code of conduct & 0.87*** & 0.06 & 14.21 \\
 & ENVRTX15 & Other compliance or auditing program focused on your plant (not on your suppliers) & 0.88*** & 0.06 & 13.72 \\
 & ENVRTX12 & Seeking or maintaining ISO14001 certification & 0.85*** & 0.09 & 9.73 \\
 & ENVRTX31 & Requesting that your suppliers sign a code of environmental conduct & 1.16*** & 0.09 & 12.72 \\
 & ENVRTX35 & Ensuring that suppliers comply with child labor laws & 1.12*** & 0.1 & 11.7 \\
 & ENVRTX36 & Asking suppliers to pay a “living wage” & 1.04*** & 0.09 & 11.17 \\
 & ENVRTX06 & Pollution control (scrubbing, waste treatment) & 0.76*** & 0.07 & 11.27 \\
 & EPRACX01 & Implementation of a certified environmental management system, such as ISO 14000. & 0.96*** & 0.09 & 10.3 \\
 & EPRACX02 & Implementation of internal environmental management procedures (e.g. environmental training program, internal environmental audit, newsletter). & 0.96*** & 0.08 & 12.34 \\
 & EPRACX03 & Use of cleaner technologies in the production process (e.g. abatement equipment) to reduce pollution emissions and/or resource use. & 0.98*** & 0.07 & 14.2 \\
 & EPRACX04 & Environment-friendly product design. & 1.21*** & 0.08 & 15.58 \\
 & EPRACX05 & Environmental improvement of packaging. & 1.0*** & 0.07 & 14.74 \\
 & EPRACX06 & Use of environment-friendly raw materials. & 0.99*** & 0.07 & 14.6 \\
JIT Practices & LAYOUTN01 & We have laid out the shop floor so that processes and machines are in close proximity to each other. & 0.71*** & 0.06 & 11.66 \\
 & LAYOUTN02 & The layout of our shop floor facilitates low inventories and fast throughput. & 0.79*** & 0.07 & 12.04 \\
 & LAYOUTN03 & Our processes are located close together, so that material handling and part storage are minimized. & 0.88*** & 0.07 & 11.87 \\
 & LAYOUTN04 & We have located our machines to support JIT production flow. & 1.03*** & 0.08 & 13.77 \\
 & JITDELN01 & Our suppliers deliver to us on a just-in-time basis. & 1.09*** & 0.09 & 12.74 \\
 & JITDELN02 & We receive daily shipments from most suppliers. & 0.8*** & 0.09 & 9.13 \\
 & JITDELN03 & Our suppliers are linked with us by a pull system. & 1.1*** & 0.09 & 12.2 \\
 & KANBANN01 & Suppliers fill our kanban containers, rather than filling purchase orders. & 0.73*** & 0.09 & 8.39 \\
 & KANBANN02 & We use a kanban pull system for production control. & 1.05*** & 0.09 & 11.34 \\
 & KANBANN03 & We use kanban squares, containers or signals for production control. & 1.08*** & 0.09 & 11.56 \\
 & LINKCN01 & Our customers receive just-in-time deliveries from us. & 1.04*** & 0.08 & 12.82 \\
 & LINKCN02 & We always deliver on time to our customers. & 0.71*** & 0.06 & 10.97 \\
 & LINKCN03 & We can adapt our production schedule to sudden production stoppages by our customers. & 0.77*** & 0.07 & 11.39 \\
 & LINKCN04 & Our customers have a pull type link with us. & 1.18*** & 0.09 & 12.66 \\
 & LINKCN05 & Our customers are linked with us via JIT systems. & 1.24*** & 0.09 & 13.44 \\
 & SCHEDN01 & We usually meet the production schedule each day. & 0.75*** & 0.06 & 12.46 \\
 & SCHEDN02 & We usually complete our daily schedule as planned. & 0.68*** & 0.05 & 12.59 \\
 & SETUPN01 & We are aggressively working to lower setup times in our plant. & 0.76*** & 0.07 & 10.88 \\
 & SETUPN02 & We have low setup times of equipment in our plant. & 0.81*** & 0.07 & 11.53 \\
 & SETUPN03 & Our workers practice setups, in order to reduce the time required. & 1.04*** & 0.09 & 11.84 \\
Environmental Performance & EPERFX01 & Overall environmental performance. & 0.83*** & 0.06 & 14.96 \\
 & EPERFX02 & Raw materials consumption. & 0.77*** & 0.05 & 14.78 \\
 & EPERFX03 & Energy consumption. & 0.96*** & 0.06 & 16.74 \\
 & EPERFX04 & Water consumption. & 0.94*** & 0.06 & 17.02 \\
 & EPERFX05 & Emissions to air. & 0.89*** & 0.06 & 15.69 \\
 & EPERFX06 & Releases to water. & 0.81*** & 0.06 & 14.38 \\
 & EPERFX07 & Solid waste generation (e.g. landfill capacity consumed). & 0.7*** & 0.05 & 13.53 \\
 & EPERFX08 & Waste recovery (e.g. recycling). & 0.59*** & 0.05 & 11.7 \\
 & EPERFX09 & Fines or other violations of environmental rules/regulations. & 0.84*** & 0.07 & 11.57 \\
\end{longtable}

\end{landscape}
\subsection*{Practice Adoption}
Preliminary Discussion:

- Staying with \cite{furlanComplementarityLeanManufacturing2011}, we have created dummy variables for the study as seen in Table 2.

- We will need to discuss how these dummy variables were created (splitting at the median point for high/low and then expanding to 4 categories based on the binary combinations).

- We still have not run a t-test to confirm statistically significant means between dummy variables as per \cite{furlanComplementarityLeanManufacturing2011}.

\subsection*{Tukey Analysis}
Preliminary Discussion:

- The Tukey analysis shows that the means of environmental performance differences are not statistically significant in certain cases.

- This fact seems to already reject H0 as we have no difference in performance between the combination and the singular (mainly environmental).

- For this reason, we probably do not need to run the OLS regression for complementarity as we can already see the marginal performance difference is not present.

- What can we do to validate the assumptions of this modeling?

\begin{table}[htbp]
\centering
\caption{Multiple Comparison of Means - Tukey HSD, FWER=0.05}
\label{tab:your_label}
\begin{tabular}{llrrrrr}
\toprule
group1 & group2 & p-adj & lower & upper & reject \\
\midrule
High JIT \& Environmental & Low JIT \& Environmental & 0.00 & -0.85 & -0.28 & True \\
High JIT \& Environmental & Mainly Environmental & 0.69 & -0.43 & 0.18 & False \\
High JIT \& Environmental & Mainly JIT & 0.00 & -0.65 & -0.14 & True \\
Low JIT \& Environmental & Mainly Environmental & 0.01 & 0.09 & 0.79 & True \\
Low JIT \& Environmental & Mainly JIT & 0.45 & -0.13 & 0.48 & False \\
Mainly Environmental & Mainly JIT & 0.15 & -0.59 & 0.06 & False \\
\bottomrule
\end{tabular}\end{table}
\subsection*{Regression Models}
Preliminary Discussion:

- For our regression models, we have taken means of our CFA bundles (JIT and Env) and used them as the independent variables.

- Again, we need to understand what the most important assumptions for this model are and how to validate them.

- So far (controlling for plant size), we see no statistically significant results for the regression models besides environmental practices as correlated to emissions to air.

- These results are surprising and make us question the validity of our models.

- Is the reason for these results the fact that we bundle over 40 practices, and the dependent variables are too singular?

- How can we diagnose and improve our models?

- General comment for all models: we have not checked the 6 assumptions of linear models.

- Where is most imporant to check for: normality, homoscedasticity, multicollinearity, autocorrelation, and linearity?


