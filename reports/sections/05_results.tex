\section{Results}
\subsection*{Complimentarity Analysis of JIT \& Environmental Practices}
First we run an ANOVA to test for statistically significant difference between the mean environmental performance outcomes of our four categories. 
First we checked for a statistically significant difference, then we a Tukey HSD test to determine which categories are significantly different from each other. 
See table 9 below for the results of our complimentarity test for the general environmental practices and JIT bundles.
\begin{table}[htbp]
\centering
\caption{Multiple Comparison of Means - Tukey HSD, FWER=0.05 (Env Practices 1 - General)}
\label{tab:your_label}
\begin{tabular}{llrrrrr}
\toprule
group1 & group2 & p-adj & lower & upper & reject \\
\midrule
High JIT \& Environmental & Low JIT \& Environmental & 0.00 & -0.85 & -0.30 & True \\
High JIT \& Environmental & Mainly Environmental & 0.20 & -0.44 & 0.06 & False \\
High JIT \& Environmental & Mainly JIT & 0.22 & -0.60 & 0.09 & False \\
Low JIT \& Environmental & Mainly Environmental & 0.00 & 0.09 & 0.67 & True \\
Low JIT \& Environmental & Mainly JIT & 0.12 & -0.05 & 0.69 & False \\
Mainly Environmental & Mainly JIT & 0.97 & -0.42 & 0.29 & False \\
\bottomrule
\end{tabular}\end{table} \\ 
From the results we can see that there is no statistically significant difference between the means for High JIT \& High Env Practices and Mainly Environmental.
There is also no statistically significant difference between the means for High JIT \& Environmental and Mainly JIT.
Based on these results we can conclude that there is no marginal gains from the combination of JIT and environmental practices on our environmental performance bundle.
It is worth noting that our environmental practices bundle was limited during the EFA to the top 8 with the highest loadings, despite this fact we similar results when running the same analysis on the full set of environmental peformance outcomes (see appendix). 
\\
For our second environmental practices bundle, focused on suppliers activity there was no statistically significant differences among categories based on ANOVA (p<0.05), so we did not run a Tukey HSD test.
\subsection*{Moderating Effect on Specific Environmental Outcomes}
Using mean of environmental performance outcomes as a dependent variable, we run a regression analysis to test for the moderating effect of JIT on the relationship between environmental practices and environmental performance.
The results are presented in the tables below.
\begin{table}[htbp]
    \centering
    \caption{Emissions to Air - Regression Results (Environmental Practices 1 - General)}
    \label{tab:regression}
    \begin{tabular}{lccccccc}
\toprule
Coefficient & Coef. & Std.Err. & t & P>|t| & [0.025 & 0.975] & Sig. \\
\midrule
Intercept & 2.28 & 2.03 & 1.12 & 0.26 & -1.73 & 6.29 &  \\
Env\_Score & 0.27 & 0.51 & 0.53 & 0.60 & -0.74 & 1.28 &  \\
JIT\_Score & -0.05 & 0.59 & -0.08 & 0.93 & -1.21 & 1.11 &  \\
JIT\_Env\_Interaction & 0.04 & 0.15 & 0.25 & 0.80 & -0.25 & 0.32 &  \\
ACCTGX51 & -0.00 & 0.00 & -0.05 & 0.96 & -0.00 & 0.00 &  \\
\bottomrule
\end{tabular}

    \end{table}
    
\begin{table}[htbp]
    \centering
    \caption{Solid Waste Generation - Regression Results (Environmental Practices 1 - General)}
    \label{tab:regression}
    \begin{tabular}{lccccccc}
\toprule
Coefficient & Coef. & Std.Err. & t & P>|t| & [0.025 & 0.975] & Sig. \\
\midrule
Intercept & 1.89 & 1.80 & 1.05 & 0.29 & -1.66 & 5.44 &  \\
Env\_Score & 0.32 & 0.45 & 0.70 & 0.48 & -0.58 & 1.21 &  \\
JIT\_Score & 0.23 & 0.52 & 0.45 & 0.65 & -0.79 & 1.26 &  \\
JIT\_Env\_Interaction & -0.02 & 0.13 & -0.16 & 0.87 & -0.28 & 0.23 &  \\
ACCTGX51 & 0.00 & 0.00 & 1.86 & 0.07 & -0.00 & 0.00 & * \\
\bottomrule
\end{tabular}

    \end{table}
    
It is clear from the results that there is no statistically significant moderating effect of JIT on the relationship between environmental practices and environmental performance for the general environmental practices bundle.
\\
Below we will also present the results of the regression analysis for the supplier orientated environmental practices bundle.
\begin{table}[htbp]
    \centering
    \caption{Emissions to Air - Regression Results (Environmental Practices 2 - Suppliers)}
    \label{tab:regression}
    \begin{tabular}{lccccccc}
\toprule
Coefficient & Coef. & Std.Err. & t & P>|t| & [0.025 & 0.975] & Sig. \\
\midrule
Intercept & 2.28 & 2.03 & 1.12 & 0.26 & -1.73 & 6.29 &  \\
Env\_Score & 0.27 & 0.51 & 0.53 & 0.60 & -0.74 & 1.28 &  \\
JIT\_Score & -0.05 & 0.59 & -0.08 & 0.93 & -1.21 & 1.11 &  \\
JIT\_Env\_Interaction & 0.04 & 0.15 & 0.25 & 0.80 & -0.25 & 0.32 &  \\
ACCTGX51 & -0.00 & 0.00 & -0.05 & 0.96 & -0.00 & 0.00 &  \\
\bottomrule
\end{tabular}

    \end{table}
    
\begin{table}[htbp]
    \centering
    \caption{Solid Waste Generation - Regression Results (Environmental Practices 2 - Suppliers)}
    \label{tab:regression}
    \begin{tabular}{lccccccc}
\toprule
Coefficient & Coef. & Std.Err. & t & P>|t| & [0.025 & 0.975] & Sig. \\
\midrule
Intercept & 1.89 & 1.80 & 1.05 & 0.29 & -1.66 & 5.44 &  \\
Env\_Score & 0.32 & 0.45 & 0.70 & 0.48 & -0.58 & 1.21 &  \\
JIT\_Score & 0.23 & 0.52 & 0.45 & 0.65 & -0.79 & 1.26 &  \\
JIT\_Env\_Interaction & -0.02 & 0.13 & -0.16 & 0.87 & -0.28 & 0.23 &  \\
ACCTGX51 & 0.00 & 0.00 & 1.86 & 0.07 & -0.00 & 0.00 & * \\
\bottomrule
\end{tabular}

    \end{table}
    
Again, there is no statistically significant moderating effect of JIT on the relationship between environmental practices and environmental performance for the supplier orientated environmental practices bundle.
It is worth noting that we do have a statistically significant result at p < 0.05 for our control variable of plant size in the regression analysis for solid waste generation in the supplier orientated environmental practices bundle.