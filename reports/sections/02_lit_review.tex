\section{Environmental Practices}
A key concept in our research is environmental practices. 
In academic literature, ‘environmental practices’ is used to describe a wide range of different environmental practices \citep{montabonExaminationCorporateReporting2007}. 
However, environmental practices are highly context-related and in our paper we will focus specifically on environmental practices in a manufacturing plant context. 
There are a lot of environmental practices present in the literature and generating an exhaustive list would be impossible. 
Separating environmental practices from non-environmental practices is not simple as many conventional practices can be perceived as environmental practices depending on the context.

There are also various other terms and concepts that are closely related to environmental practices depending on the context. 
Many articles include discussion about environmental management practices which can be seen as environmental practices. 
The term ‘green practices’ is also occasionally mentioned and it is often interchangeable with the concept of environmental practices.

For the purpose of examining environmental practices, researchers have developed multiple sets of environmental practices that are used in the papers for analysis and surveys \citep{montabonExaminationCorporateReporting2007, zhuRelationshipsOperationalPractices2004}. 
In the literature, different categorizations for environmental practices have been displayed. 
Montabon et al. (2007) divided their list of environmental practices used in their study into operational, tactical and strategic practices. 
This was done in order to recognize that different practices have different scopes and impacts \citep{montabonExaminationCorporateReporting2007}. 
In our research, we will consider environmental practices comprehensively and use a list of environmental practices developed by the HPM survey.
\section{Lean practices and JIT delivery}
\subsection{Lean as a concept}
Lean manufacturing refers to manufacturing where lean has been implemented. 
There are a lot of different definitions for lean in academic literature \citep{sundarReviewLeanManufacturing2014}. 
Lean is often thought of as activities relating to waste reduction, but in practice the fundamental purpose of lean is to increase the value of output by reducing waste in production processes \citep{sundarReviewLeanManufacturing2014}. 
For this paper, lean is defined loosely as a system that aims at continuous improvement and elimination of all kinds of waste \citep{simpsonUseSupplyRelationship2005}. 
Lean practices include for example Just-in-Time manufacturing, Kanban, value stream mapping, and push and pull systems \citep{sundarReviewLeanManufacturing2014}. 
The main benefits of lean relate to increased productivity and quality while costs are reduced \citep{bhamuLeanManufacturingLiterature2014}.

Lean manufacturing is a holistic way of working. 
\citep{kingLeanGreenEmpirical2009} point out that lean manufacturing includes numerous practices, which spread over the entire scope of the organisation. 
Lean manufacturing can thus be seen to cover aspects of product development, operations management, supply chain management, design and manufacturing \citep{bhamuLeanManufacturingLiterature2014}. 
\citep{sundarReviewLeanManufacturing2014} add that lean implementation requires a proper sequencing and integration plan. 
For example, cultural change and employee training on lean concepts are needed to make the implementation successful.

\subsection{Just-in-time delivery}
In the literature, the concepts of lean and Just-In-Time are intertwined and many, such as \citep{bhamuLeanManufacturingLiterature2014}, \citep{belekoukiasImpactLeanMethods2014} recognize the close connection between them. 
Just-in-Time is a critical aspect of lean manufacturing practices, focusing on the efficiency of production timing and inventory management.

“Lean manufacturing has been widely implemented by manufacturing organisations to achieve operational excellence, and in this way meet both traditional and contemporary organisational objectives such as profitability, efficiency, responsiveness, quality and customer satisfaction” \citep{garza-reyesLeanGreenSystematic2015}. 
“JIT is based on producing the right goods at the right time” \citep{womackLeanThinkingBanish1997}. 
This contributes in reducing space utilisation, inventory and wastes associated to the overproduction of goods.

\section{Environmental performance}
Environmental performance means how well an organisation performs in relation to its environmental responsibilities \citep{mollenkopfGreenLeanGlobal2010}. 
It is measured by how much natural resources have been consumed and by how much waste of water, gases and poisonous materials are emitted \citep{maoLowCarbonSupply2017}. 
In this paper, the focus is on studying the environmental performance of manufacturing plants. 
We do not consider how the manufacturing plant contributes to the environmental performance of the whole supply chain. 

% Subsections for CO2 emissions and Packaging waste

\section{Relationship between JIT and environmental performance}
Most of the existing literature states that lean practices used to decrease the environmental impact of a company are successful \citep{diesteEvaluatingImpactLean2020}. 
However, there are different opinions among scholars on whether lean practices have a positive impact on environmental performance and lean practices can according to the literature have both positive and negative impacts on environmental performance \citep{diesteEvaluatingImpactLean2020}.

According to the research conducted by \citep{diesteEvaluatingImpactLean2020}, the general trend seems to be that lean practices improve long term environmental performance. 
Lean processes can aid companies in achieving their environmental goals if they are committed to the goals and aware of the organisation’s environmental impact \citep{diesteEvaluatingImpactLean2020}. 
Some researchers state that lean companies can improve their environmental performance since the lean practices also focus on waste reduction and process efficiency \citep{diesteEvaluatingImpactLean2020}. 
More specifically, JIT practices can improve the environmental supply chain performance \citep{cherrafiLeanGreenPractices2018, diesteEvaluatingImpactLean2020} and decrease the fuel consumption since smaller vehicles can be used for smaller deliveries \citep{garza-reyesLeanGreenTransport2016}. 
Further, JIT can reduce energy consumption of storage since it reduces the inventory volume \citep{garza-reyesEffectLeanMethods2018}.

Additionally, since lean practices assert waste reduction, they can naturally lead to better environmental practices and an internal environment that supports the adaptation of these practices \citep{garza-reyesEffectLeanMethods2018}. 
Further, aspects worth considering are that lean companies more likely adapt environmental innovations \citep{mollenkopfGreenLeanGlobal2010, garza-reyesEffectLeanMethods2018}. 

However, being more productive and efficient in manufacturing does not equal more environmental sustainability and several papers address negative or mixed impacts of lean on air emissions, energy use and water use \citep{diesteEvaluatingImpactLean2020}. 
Out of the lean practices, JIT is the most problematic due to its nature of small deliveries which can increase additional waste and emissions \citep{rothenbergLEANGREENQUEST2009, Venkat_Wakeland_2006, diesteEvaluatingImpactLean2020} and some scholars argue that JIT and positive environmental performance cannot be combined \citep{zhuRelationshipsOperationalPractices2004, diesteEvaluatingImpactLean2020}. 
To specify, according to \citep{sartalAreAllLean2018, diesteEvaluatingImpactLean2020}, the larger amount of JIT processes at the plant, the worse the environmental impact. 
Even if JIT can have positive inventory effects, its effects on pollution are especially debated in existing literature \citep{garza-reyesEffectLeanMethods2018}. 
To further specify, recurrent deliveries increase the transportation need, which in turn increases the air emissions \citep{diesteEvaluatingImpactLean2020}.

Moreover, \citep{garza-reyesEffectLeanMethods2018} point out that most of the previous research conducted has focused on very specific lean practices and the environmental measures have varied significantly between studies. 
Through this, they argue that how lean practices affect environmental performance can still be labelled as inconclusive \citep{garza-reyesEffectLeanMethods2018}. 
\citep{garza-reyesEffectLeanMethods2018} call for further research regarding the effect of lean manufacturing practices on environmental performance in other industrial sectors.
Building on this literature review of existing research, the research questions for this study are the following:\\

\textbf{RQ 1:} What effect do lean JIT practices have on environmental practices and environmental performance?\\

\textbf{RQ 2:} What is the effect of JIT practices on toxic air emissions and solid waste generation?
